\documentclass[reqno, 12pt]{amsart}

\usepackage{amssymb}
\usepackage{amsthm}
\usepackage{amsmath}
\usepackage{amsxtra}
\usepackage{mathrsfs}
%\usepackage{enumerate}
\usepackage[inline, shortlabels]{enumitem}
\usepackage{comment}
\usepackage{graphicx}
%\usepackage[all]{xy}
\usepackage{placeins}
\usepackage{multicol}
\usepackage{bbm}
\usepackage{cancel}
\usepackage{fullpage}
\usepackage{stmaryrd}
\usepackage{subcaption}
%\usepackage{calc}
%\usepackage{tikz}
%\usetikzlibrary{arrows,calc,automata,shadows,backgrounds,positioning,intersections,fadings,decorations.pathreplacing,shapes,snakes, matrix}
\usepackage{tikz-cd}

%\usepackage{listings}
%\usepackage{color}
%
%\definecolor{dkgreen}{rgb}{0,0.6,0}
%\definecolor{gray}{rgb}{0.5,0.5,0.5}
%\definecolor{mauve}{rgb}{0.58,0,0.82}
%
%\lstset{frame=tb,
%  language=Python,
%  aboveskip=3mm,
%  belowskip=3mm,
%  showstringspaces=false,
%  columns=flexible,
%  basicstyle={\small\ttfamily},
%  numbers=none,
%  numberstyle=\tiny\color{gray},
%  keywordstyle=\color{blue},
%  commentstyle=\color{dkgreen},
%  stringstyle=\color{mauve},
%  breaklines=true,
%  breakatwhitespace=true,
%  tabsize=3
%}

\setlength{\hfuzz}{4pt}

\newtheorem{thm}{Theorem}%[section]
\newtheorem{lem}{Lemma}
\newtheorem{cor}[thm]{Corollary}
\newtheorem{conj}[thm]{Conjecture}
\newtheorem{prop}[thm]{Proposition}
\theoremstyle{definition}  
\newtheorem{defn} [thm] {Definition} 
\newtheorem{claim}[thm]{Claim}
\newtheorem{example} [thm] {Example}
\newtheorem{rem} [thm] {Remark}

\newcommand{\C}{\mathbb C}
\newcommand{\D}{\mathbf D}
\renewcommand{\H}{\mathbf H}
\newcommand{\F}{\mathbb F}
\newcommand{\Q}{\mathbb Q}
\newcommand{\R}{\mathbb R}
\newcommand{\Z}{\mathbb Z}
\newcommand{\T}{\mathbb T}
\renewcommand{\P}{\mathbb P}
\renewcommand{\O}{\mathcal O}
\newcommand{\M}{\mathfrak M}
\newcommand{\A}{\mathbb A}
\newcommand{\V}{\mathbb V}
\newcommand{\I}{\mathbb I}
\newcommand{\m}{\mathfrak m}
\renewcommand{\SS}{\mathbb S}

\renewcommand{\L}{\mathcal{L}}

\renewcommand{\Re}{\text{Re}}
\renewcommand{\Im}{\text{Im}}

\newcommand{\gl}{\mathfrak gl}
\renewcommand{\sl}{\mathfrak sl}
\newcommand{\s}{\mathfrak s}
\renewcommand{\a}{\mathfrak a}
\newcommand{\p}{\mathfrak p}
\newcommand{\q}{\mathfrak q}
\newcommand{\normal}{\trianglelefteq}

\newcommand{\charac}{\text{char}}
\newcommand{\la}{\langle}
\newcommand{\ra}{\rangle}
\newcommand{\wt}{\widetilde}

\newcommand{\Kbar}{\overline{K}}

\renewcommand{\div}{\text{div}}

\DeclareMathOperator{\lcm}{lcm}
\DeclareMathOperator{\img}{img}
\DeclareMathOperator{\Tor}{Tor}
\DeclareMathOperator{\tr}{tr}
\DeclareMathOperator{\Hom}{Hom}
\DeclareMathOperator{\End}{End}
\DeclareMathOperator{\Aut}{Aut}
\DeclareMathOperator{\Gal}{Gal}
\DeclareMathOperator{\sgn}{sgn}
\DeclareMathOperator{\ord}{ord}
\DeclareMathOperator{\Ob}{Ob}
\DeclareMathOperator{\genus}{genus}
%\DeclareMathOperator{\GL}{GL}
%\DeclareMathOperator{\SL}{SL}
%\DeclareMathOperator{\SO}{SO}
%\DeclareMathOperator{\Sp}{Sp}
\DeclareMathOperator{\Spec}{Spec}
\DeclareMathOperator{\Stab}{Stab}
\DeclareMathOperator{\Frac}{Frac}
\DeclareMathOperator{\Div}{Div}
\DeclareMathOperator{\Pic}{Pic}
\DeclareMathOperator{\supp}{supp}
%\DeclareMathOperator{\wasy}{\wasylozenge}
\DeclareMathOperator{\Frob}{Frob}
\DeclareMathOperator{\Nm}{Nm}
\DeclareMathOperator{\rank}{rank}

\DeclareMathOperator{\trd}{trd}
\DeclareMathOperator{\nrd}{nrd}

\DeclareMathOperator{\Supp}{Supp}
\DeclareMathOperator{\res}{res}

\DeclareMathOperator{\id}{id}

\newcommand{\sss}[1]{{\color{blue} [#1]}}
\newcommand{\jh}[1]{{\color{green} [#1]}}


%\newcommand{\res}[3]{\left(\frac{#1}{#2} \right)_{#3}}

\renewcommand{\thesection}{\Roman{section}} 

\usepackage{mathpazo}
%\usepackage{pxfonts}

\everymath{\displaystyle}
\tikzset{commutative diagrams/.cd, arrow style = tikz, diagrams = {>=latex}}


\begin{document}

\title{Glueing Jacobians}
%\author{Alex Levin and Sam Schiavone}
\author{Jeroen Hanselman, Sam Schiavone, and Jeroen Sijsling}
%\date{30 June 2017}
\date{\today}

\maketitle

\tableofcontents

\section{Introduction}
In \cite{FreyKani}
%Frey and Kani's paper \textit{Curves of genus 2 covering elliptic curves and an arithmetical application},
the authors describe a method for glueing two elliptic curves $E_1$ and $E_2$ along their torsion subgroups to produce a genus $2$ curve that covers both of them. In this article, we extend this method to genus $3$: we glue a genus $1$ curve $X_1$ to the Jacobian variety of a genus $2$ curve $X_2$. This produces an abelian $3$-fold which, since all abelian $3$-folds are principally polarized, is the Jacobian variety of a genus $3$ curve $X_3$. We determine explicit equations for $X_3$, given the data of blah. We have implemented this method in \textsf{Magma}, and conclude the paper with several examples.

\sss{Other papers to mention? Howe? Howe, Leprovost, Poonen? Broker, Lauter, Stevenhagen, etc.?}

\section{Background}

Encoding divisors as polynomials as in Mumford and Cantor. Describe construction of the Kummer as in Mueller.

\subsection{Representing divisors}

\subsection{Embedding the Kummer variety}

\section{Overview of method}

Our construction proceeds as follows. We take as input an elliptic curve $X_1$ and a genus $2$ curve $X_2$ over a number field $K$ \sss{or more genenerally, any field?} given in Weierstrass form
$$
X_1: y^2 + u(x) y = v(x) \qquad
X_2: y^2 + h(x) y = f(x) \, .
$$
Letting $J_2$ be the Jacobian variety of $X_2$, then $J_2$ is an abelian surface with $16$ $2$-torsion points. The Kummer variety $K_2$ of $X_2$ is obtained by forming the quotient of $J_2$ by the negation map $[-1]$. This quotient map $\pi : J_2 \to K_2$ is injective on the $2$-torsion points of $J_2$, whose images are the singular points of $K_2$. \sss{Nodes, I guess?} Using  the explicit embedding given in \cite{Mueller} (which in turn is a generalization of \cite{CasselsFlynn}), we can realize $K_2$ as a quartic surface in $\P^3$.

Fix two nodes $T_1, T_2$ of $K_2$. Consider the pencil of planes $\mathcal{H} = \{H_\mu : \mu \in \P^1\}$ passing through $T_1$ and $T_2$. The intersection of a plane $H_\mu \in \mathcal{H}$ with $K_2$ is a quartic plane curve $C_\mu$ with two nodes. By the usual degree-genus formula for plane curves, $C_\mu$ has genus $1$ for each $\mu \in \P^1$. We will endow $C_\mu$ with the structure of an elliptic curve and compute its $j$-invariant as a function of $\mu$.

To a point $Q \in C_\mu$ we associate the line $\ell_Q$ passing through $T_1$ and $Q$. The association $Q \mapsto \ell_Q$ defines a degree $2$ map $C_\mu \to \P^1$ ramified at 4 points. Computing the cross-ratio of these 4 points yields the $\lambda$-invariant of $C_\mu$, allowing us to find a Legendre model $y^2 = x(x-1)(x-\lambda)$ for $C_\mu$. We can then compute the $j$-invariant of $C_\mu$ using the standard formula
$$
j = 2^8 \frac{(\lambda^2 - \lambda + 1)^3}{\lambda^2 (\lambda-1)^2}  \, .
$$
Note that computing the $\lambda$-invariant of $C_\mu$ not only endows $C_\mu$ with the structure of an elliptic curve, but also with level $2$ structure: the Legendre model $E_\text{Leg}: y^2 = x(x-1)(x-\lambda)$ comes equipped with the basis $\{(0,0), (1,0)\}$ for $E_\text{Leg}[2]$, and we may pull back this basis along the isomorphism $C_\mu \overset{\sim}{\to} E_\text{Leg}$ to obtain a basis for $C_\mu[2]$.
\begin{lem}
The composite map
%\begin{align*}
%\P^1 &\to \mathcal{M}_1 \to X(2) \to X(1)\\
%
%\end{align*}
$$
\begin{tikzcd}[row sep = small]
\varphi: \P^1 \ar[]{r}{} & \mathcal{M}_1 \ar[]{r}{} & X(2) \ar[]{r}{} & X(1)\\ 
\mu \ar[mapsto]{r}{} & C_\mu \ar[mapsto]{r}{} & \lambda(C_\mu) \ar[mapsto]{r}{} & j(\lambda(C_\mu))
\end{tikzcd}
$$
has degree $12$.
\end{lem}

\begin{proof}
By the classical theory of modular functions, the map $X(2) \to X(1)$, $\lambda \mapsto j(\lambda)$ has degree $6$, corresponding to the $6$ permutations of $0, 1, \infty$ acted on by $S_3$. As the map $\mu \mapsto C_\mu$ has degree $1$, it suffices to show that the map $\mathcal{M}_1 \to X(2)$ has degree $2$. \sss{I think this just follows from the fact that we could've chosen to the other node and taken lines through $T_2$ and $Q$ to obtain a map to $\P^1$. I guess we have to show that this would produce the same $\lambda$...}
\end{proof}

Thus the composite map in the above lemma is a rational function of degree $12$ in $\mu$. Let $j_1 = j(X_1)$. In order to find a value of $\mu$ that yields an elliptic curve $C_\mu$ isomorphic to our original curve $X_1$, we solve the equation $\varphi(\mu) = j_1$. The solutions $\mu$ to this equation may not lie in the ground field, so it may be necessary to base change our curve to an algebraic extension. \sss{I think in all the examples so far we've only needed quadratic extensions of the base field...} \sss{One more interesting note: I think in all the examples we've done so far, $\varphi(\mu) - j_1$ has an interesting factorization. The numerator is a product of quadratics, and the denominator is a product of linear factors squared. Is this expected?}

\section{Applications}
Constructing abelian three-folds with interesting torsion?

\section{Examples}

%\nocite{*} % use this to have all references listed, not just those cited
%\bibliographystyle{amsplain}
\bibliographystyle{alpha}
%\addcontentsline{toc}{chapter}{References}
\bibliography{references}


\end{document} 
